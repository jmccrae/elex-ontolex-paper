\documentclass[12pt,a4paper]{elex2017}
\usepackage[left=2.5cm,right=2.5cm,bottom=2.5cm,top=2.5cm]{geometry}
\usepackage[english]{babel}
\usepackage[T1]{fontenc}
\usepackage[utf8]{inputenc}
\usepackage[unicode=true]{hyperref}
\usepackage{graphicx}
\usepackage[
    style=authoryear,
    natbib=true,
    defernumbers=true
]{biblatex}
\addbibresource{elex-ontolex.bib}

\setcounter{secnumdepth}{3}
\let\subparagraph\paragraph
%\let\subparagraph\relax

% numbering of sections, format of the title
\makeatletter
% we use \prefix@<level> only if it is defined
\renewcommand{\@seccntformat}[1]{%
  \ifcsname prefix@#1\endcsname
    \csname prefix@#1\endcsname
  \else
    \csname the#1\endcsname\quad
  \fi}
\newcommand\prefix@section{\thesection. }
\makeatother

\setlength{\parindent}{0cm}
\setlength{\parskip}{11pt plus 1pt minus 2pt}
\setlength{\bibhang}{1cm}
\setlength{\baselineskip}{16pt}

%\pagestyle{empty}

%%%%%%%%%%%%%%%%%%%%%%%%%%%%%%%%%%%%%%%%%%%%%%%%%%%%%%%%%%%%%%%%%%%%%%%%%%%%%%%%
% DOCUMENT BODY
%%%%%%%%%%%%%%%%%%%%%%%%%%%%%%%%%%%%%%%%%%%%%%%%%%%%%%%%%%%%%%%%%%%%%%%%%%%%%%%%

\begin{document}
\mainmatter
\title{The OntoLex-Lemon Model: development and applications}
\titlerunning{The OntoLex-Lemon Model}
\author{\bf John P. McCrae$^1$, Paul Buitelaar$^1$, Philipp Cimiano$^{2}$}
\institute{$^1$Insight Centre for Data Analytics, National University of Ireland
Galway\\ $^2$ Cognitive Interaction Technology Excellence Cluster, Bielefeld
University\\
E-mail: john@mccr.ae, paul.buitelaar@insight-centre.org,
cimiano@cit-ec.uni-bielefeld.de}
\toctitle{The OntoLex-Lemon Model: development and applications}

\maketitle

\begin{abstract}
Each article must include an abstract of 150 to 200 words in Latin Modern Roman
11 pt with interlinear spacing of 12 pt. The heading Abstract should be
centered, font Times 10 bold. This short abstract will also be used for
printing a Booklet of Abstracts containing the abstracts of all papers
    presented at the Conference.

\keywords{provide 3--5 keywords, separated by semi-colons}
\end{abstract}


% 25,000-50,000 Charcters = 8-15 pages

%The lemon Model, first proposed in~\cite{mccrae2012interchanging}, has become the primary
%model for the representation of lexical data on the Semantic Web and has been
%further developed in the context of the W3C OntoLex Community Group. In this
%paper we will present the development and future outlooks for this model as well
%as briefly detailing some of the current applications of the model. After the
%lemon model was developed in the context of the Monnet project, it was decided
%that the further development of this model should take place within a forum as
%open as possible, which fortunately coincided with the creation of community
%groups for W3C. The community group structure provided mailing lists and wikis
%for discussion of the model and eventually led to the publishing of the model as
%a W3C Report (\cite{cimiano2016lexicon}) and as files in the W3C namespace. The
%discussions within the group covered all aspects of the model, however the issue
%of semantics was of particular interest to the group and led to a major
%innovation in the introduction of a lexical concept, as a distinct element from
%the ontology reference. The formal distinction between these is that an ontology
%reference is an entity in an ontology, which the word denotes.  As such, the
%(ontological) meaning of  the question “When did Prince die?” could be
%understood with the ontology predicate deathDate as a reference, but the general
%concept of dying refers to an event rather than to a date. This also further
%extends the application domain of OntoLex-Lemon, from formal applications such
%as question answering and semantic parsing to the representation of general
%machine-readable dictionaries, including WordNet and digitized versions of
%existing dictionaries.
%Thus, the OntoLex-Lemon model has continued to expand in its use cases and we
%consider two particular cases here. Firstly, the OntoLex-Lemon model has been
%adopted in a variety of online dictionaries and this has provided a common
%interface to these dictionaries. This can be exploited to provide a single
%access point across multiple dictionaries that is implemented in the back-end by
%means of generic SPARQL (a Semantic Web version of SQL) queries. Secondly, we
%look at the application of OntoLex-Lemon in the context of the WordNet
%Collaborative Interlingual Index (\cite{bond2016cili}), where the model is being
%used to provide a single interlingual identifier for every concept in every
%language. The application of ontolex-lemon to these two cases will be discussed
%in detail at the eLex conference and in a final paper if our contribution is
%accepted.
%Finally, we consider the future of the OntoLex-Lemon model, which we intend to
%continue to develop and have recently identified four areas for extension:
%increasing support for use cases of the model in representing digitized
%dictionaries, the use of clear and defined data categories to improve
%interoperability, the development of a module for representing complex
%morphological patterns and finally an extension to support the representation of
%diachronic and etymological information within the lexicon. These developments
%should further increase the applicability and value of the model to more users.

\section{Introduction}

Ontologies have become an increasingly important way of modelling domains and
representing data in a variety of forms, most notably the Semantic Web. However
the existing standards for ontologies, most notably the Web Ontology
Language~\cite[OWL]{mcguinness2004ow}, provide little support for the
representing information about how a word is expressed in language, beyond a
simple string. In order to close this gap, the \emph{lemon}
Model~\cite{mccrae2012interchanging} was proposed, which created a separate
lexicon that could describe how an ontological concept was lexicalized in more
details. This builds on the paradigm of the ontology-lexicon interface, whereby
the link between how a concept is expressed in natural language and the formal
description of the concept in the ontology is kept separated. This has several
advantages, most notably in that by separating the ontological and the lexical
layer we can easily switch an ontology from one language to another by changing
its lexicon. 

The \emph{lemon} Model was adopted by a number of projects~\cite{todo} and
several authors have proposed modifications, improvements or changes~\cite{todo}
to the model. In order to accommodate these changes, it was decided that the
model should be further developed under an open forum and for this purpose the
authors of this paper founded the OntoLex Community Group. This group was part
of the World Wide Web Consortium's Business and Community group program.

\section{The OntoLex Community Group}

\section{The OntoLex Model}

\section{Use cases}

\subsection{Representing dictionaries with OntoLex}

\subsection{The Collaborative Interlingual Index}

\section{Extensions and Future Plans}

\section{Conclusion}

\section*{Acknowledgement} 

Place all acknowledgements (including those concerning research grants and
funding) in a separate section at the end of the article.

\section*{References}

%\nocite{*}
\printbibliography[
    type={book},
    notkeyword={dictionary},
    title={Books}
]
\printbibliography[
    type={incollection},
    title={Book Sections}
]
\printbibliography[
    type={inproceedings},
    title={Paper in conference proceedings}
]
\printbibliography[
    type={article},
    title={Journal Articles}
]
\printbibliography[
    type={misc},
    title={Technical Reports}
]
\printbibliography[
    type={book},
    keyword={dictionary},
    title={Dictionaries}
]


\medskip
\begin{minipage}[t]{\textwidth}
    \noindent This work is licensed under the Creative Commons Attribution
    ShareAlike 4.0 International License.
    \vspace{-2ex}
    \begin{center}%
        \url{http://creativecommons.org/licenses/by-sa/4.0/}\linebreak
        \includegraphics[width=2.33cm]{cc.png}%
    \end{center}
\end{minipage}

\end{document}
% vim: noai nocin nosi inde=:
